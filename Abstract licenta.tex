\documentclass[a4paper]{article}
\usepackage{amsmath}
\usepackage{graphicx}
\usepackage{hyperref}
\usepackage[english]{babel}
\usepackage[utf8]{inputenc}

\title{Introducere}
\author{Muscă Ioana Valentina}
\date{}

\begin{document}
\maketitle


\begin{normalsize}

\hspace{5mm}Robotul urmăritor de linie (line follower) este un mecanism mobil care poate detecta și urmări o linie de pe o suprafață plană. În general, linia este de culoare neagră și este percepută de robot cu ajutorul senzorilor cu infraroșu instalați sub robot. Datele strânse de senzori sunt transmise procesorului care va decide ce traseu să urmeze și va trimite comenzile de direcție robotului.\\

\hspace{5mm}Pentru o mai bună interpretare a traseului, robotul poate fi îmbunătățit cu un senzor de obstacole ce ajuta la evitarea coleziunii cu alte obiecte în decursul traseului. Cele patru motoare ce acționează roțile se bazează pe principiul șenilelor de tanc, motoarele sunt legate în paralel iar pentru a efectua un viraj un set de motoare se oprește iar setul paralel se activează.\\

\hspace{5mm}Arduino este tehnologia utilizată pentru programarea robotului, acesta dispune de module arduino pentru setarea PWM-ului motoarelor, activarea și dezactivarea motoarelor și procesarea datelor trimise de senzori. Acest model de robot are aplicabilitate în zilele noastre în domeniul automotive, se încearcă robotizarea mașinilor pentru a putea efectua singure un traseu, încadrându-se într-o bandă de mers respectând anumite limite impuse de marcajul șoselei.\\

\end{normalsize}

\end{document}
